\newglossaryentry{glossar}{name={Glossar},description={In einem Glossar werden Fachbegriffe und Fremdwörter mit ihren Erklärungen gesammelt.}}
\newglossaryentry{glossaries}{name={Glossaries},description={Glossaries ist ein Paket was einen im Rahmen von LaTeX bei der Erstellung eines Glossar unterstützt.}}
\newglossaryentry{UGV}
{
    name=Unmanned Ground Vehicle,
    description={Ein unbemanntes Bodenfahrzeug, das autonom oder ferngesteuert betrieben wird, typischerweise für militärische oder zivile Anwendungen}
}

\newglossaryentry{UAV}
{
    name=Unmanned Aerial Vehicle,
    description={Ein unbemanntes Luftfahrzeug, das autonom oder ferngesteuert betrieben wird, oft für Aufklärungs- oder Überwachungsmissionen}
}

\newglossaryentry{CNN}
{
    name=Convolutional Neural Network,
    description={Ein tiefes neuronales Netzwerk, das speziell für die Verarbeitung von Bilddaten entwickelt wurde}
}

\newglossaryentry{IMUGS}
{
    name=Integrated Modular Unmanned Ground System,
    description={Ein Projekt zur Entwicklung modularer unbemannter Bodensysteme, das von der EU finanziert wird}
}

\newglossaryentry{SLAM}
{
    name=Simultaneous Localization and Mapping,
    description={Ein Algorithmus, der es einem mobilen Roboter ermöglicht, eine Karte seiner Umgebung zu erstellen und gleichzeitig seine Position in dieser Karte zu bestimmen}
}

\newglossaryentry{OpenCV}
{
    name=OpenCV,
    description={Eine Open-Source-Bibliothek für Computer Vision und maschinelles Lernen, die zur Bildverarbeitung und -analyse verwendet wird}
}

\newglossaryentry{AirSim}
{
    name=AirSim,
    description={Ein hochrealistischer Simulator für autonome Fahrzeuge und Drohnen, der auf der Unreal Engine basiert}
}

\newglossaryentry{Unreal Engine}
{
    name=Unreal Engine,
    description={Eine von Epic Games entwickelte Spiel-Engine, die häufig für die Entwicklung von Videospielen und Simulationen verwendet wird}
}

\newglossaryentry{BirdView}
{
    name=BirdView,
    description={Eine Perspektive, die das Gelände aus einer erhöhten, oft luftigen Position zeigt, ähnlich der Ansicht eines Vogels}
}

\newglossaryentry{Feature Matching}
{
    name=Feature Matching,
    description={Ein Verfahren in der Bildverarbeitung, bei dem markante Merkmale in verschiedenen Bildern erkannt und miteinander verglichen werden}
}
\newglossaryentry{LIDAR}
{
    name=LIDAR,
    description={Light Detection and Ranging. Ein Verfahren zur optischen Abstands- und Geschwindigkeitsmessung, das auf der Aussendung und dem Empfang von Lichtimpulsen basiert. Es wird häufig in autonomen Fahrzeugen verwendet, um die Umgebung in 3D zu erfassen}
}
