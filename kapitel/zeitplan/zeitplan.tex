\section{Zeitplan mit Meilensteinen und Risikobewertung}
Durch den vorzeitigen Beginn der Arbeit konnte zusätzliche Zeit eingeplant werden, um unvorhergesehene Verzögerungen abzufedern und die Qualität durch regelmäßige Feedbackschleifen weiter zu optimieren. \cite{ChatGPT2024}

\begin{itemize}
\item Literaturrecherche (Woche: 1 - 3) \cite{ChatGPT2024}
\begin{itemize}
\item Meilenstein 1: Sammlung und Sichtung relevanter Literatur (Ende Woche 1) \cite{ChatGPT2024}
\item Meilenstein 2: Detaillierte Analyse der wichtigsten Quellen (Ende Woche 2) \cite{ChatGPT2024}
\item Meilenstein 3: Abschluss der Literaturrecherche und Vorbereitung der theoretischen Grundlagenkapitel (Ende Woche 3) \cite{ChatGPT2024}
\item Risikobewertung: Risiko der Überschneidung oder Redundanz in der Forschungsliteratur, Gegenmaßnahmen: Frühzeitige Rücksprache mit dem Betreuer. \cite{ChatGPT2024}
\end{itemize}

\item Methodenauswahl und Experimentplanung (Woche: 4 - 5) \cite{ChatGPT2024}
\begin{itemize}
  \item Meilenstein 4: Auswahl und Begründung der Methodik, inklusive Bewertungskriterien für die Navigationsmethoden (Ende Woche 4) \cite{ChatGPT2024}
    \item Meilenstein 5: Detaillierte Planung des Experiments, Definition der zu sammelnden Daten (Ende Woche 5) \cite{ChatGPT2024}
    \item Risikobewertung: Risiko technischer Probleme bei der Einrichtung der Simulationsumgebung, Gegenmaßnahmen: Einplanung zusätzlicher Zeitpuffer und Rücksprache mit technischen Experten. \cite{ChatGPT2024}
\end{itemize}

\item Experimente durchführen (Woche: 6 - 8) \cite{ChatGPT2024}
\begin{itemize}
    \item Meilenstein 6: Aufbau und Test der Simulationsumgebung (Ende Woche 6) \cite{ChatGPT2024}
    \item Meilenstein 7: Durchführung der Experimente und Erhebung der ersten Datensätze (Ende Woche 7) \cite{ChatGPT2024}
    \item Meilenstein 8: Abschluss der Experimente, Überprüfung der Datenqualität und Datensicherung (Ende Woche 8) \cite{ChatGPT2024}
    \item Risikobewertung: Risiko unzureichender Datenqualität oder unerwarteter Ergebnisse, Gegenmaßnahmen: Flexible Anpassung des Experiments und erneute Testläufe. \cite{ChatGPT2024}
\end{itemize}

\item Datenanalyse (Woche: 9 - 10) \cite{ChatGPT2024}
\begin{itemize}
    \item Meilenstein 9: Erste Analyse und Visualisierung der Daten (Ende Woche 9) \cite{ChatGPT2024}
    \item Meilenstein 10: Interpretation der Ergebnisse im Kontext der Forschungsfragen (Ende Woche 10) \cite{ChatGPT2024}
    \item Risikobewertung: Risiko, dass die Datenanalyse längere Zeit in Anspruch nimmt als geplant, Gegenmaßnahmen: Zusätzliche Zeitreserven in der Schreibphase einplanen. \cite{ChatGPT2024}
\end{itemize}

\item Schreibphase (Woche: 11 - 13) \cite{ChatGPT2024}
\begin{itemize}
  \item Meilenstein 11: Erstellung der ersten Entwürfe für jedes Kapitel (Ende Woche 11) \cite{ChatGPT2024}
    \item Meilenstein 12: Vollständiger erster Entwurf der Bachelorarbeit, inklusive aller Kapitel und Abschnitte (Ende Woche 12) \cite{ChatGPT2024}
    \item Meilenstein 13: Überarbeitung basierend auf Betreuerfeedback und detaillierte Korrekturen (Ende Woche 13) \cite{ChatGPT2024}
    \item Risikobewertung: Risiko, dass das Feedback umfangreiche Überarbeitungen erfordert, Gegenmaßnahmen: Einarbeitung von Feedback in früheren Phasen und kontinuierliche Rücksprache mit dem Betreuer. \cite{ChatGPT2024}
\end{itemize}

\item Abschluss und Korrektur (Woche: 14 - 15) \cite{ChatGPT2024}
\begin{itemize}
    \item Meilenstein 14: Fertigstellung der Endfassung und formale Abgabe zur Korrektur (Ende Woche 14) \cite{ChatGPT2024}
    \item Meilenstein 15: Letzte Korrekturen und Abgabe der Bachelorarbeit (Ende Woche 15) \cite{ChatGPT2024}
    \item Risikobewertung: Risiko von Verzögerungen bei der formalen Abgabe, Gegenmaßnahmen: Frühzeitige Klärung der formalen Anforderungen und rechtzeitige Einreichung. \cite{ChatGPT2024}
\end{itemize}

\item Regelmäßige Rücksprache mit dem Betreuer (Woche: fortlaufend) \cite{ChatGPT2024}
\begin{itemize}
    \item Check-ins: Geplante Check-ins mit dem Betreuer nach Abschluss jedes Meilensteins, um sicherzustellen, dass die Arbeit auf Kurs bleibt. \cite{ChatGPT2024}
\end{itemize}

\end{itemize}
