\section{Nicht-Ziele}

Diese Arbeit konzentriert sich ausschließlich auf die Untersuchung der Effizienz und Robustheit bildbasierter Navigationsmethoden für \ac{UGV}s. 
Um den Fokus klar zu halten und die Reichweite der Arbeit zu definieren, werden folgende Aspekte bewusst nicht behandelt:

\begin{itemize}
  \item Datenübertragung zwischen \ac{UAV} und \ac{UGV}: 
    Diese Arbeit befasst sich nicht mit der Frage, wie die Daten vom \ac{UAV} zum \ac{UGV} übertragen werden. 
    Die technischen Details der Kommunikationsinfrastruktur zwischen \ac{UAV} und \ac{UGV} liegen außerhalb des Rahmens dieser Untersuchung und werden nicht analysiert.

  \item Sicherheitsaspekte: 
    Sicherheitsaspekte zur Vermeidung von Risiken werden nicht behandelt, da der Fokus auf der Untersuchung der bildbasierten Navigationsmethoden liegt. 
    Fragen zur Vermeidung von Kollisionen oder zur Sicherheitszertifizierung der Systeme sind nicht Teil dieser Arbeit und werden bewusst ausgeklammert.

  \item Entwicklung von Bildverarbeitungsalgorithmen: 
    Die Arbeit konzentriert sich auf die Anwendung bestehender Bildverarbeitungsalgorithmen zur Navigation und nicht auf die Entwicklung oder Optimierung neuer Algorithmen. 
    Die Leistung und Anpassung der Algorithmen wird untersucht, jedoch ohne tiefere technische Eingriffe in deren Struktur.

  \item Wirtschaftliche Analyse: 
    Eine wirtschaftliche Analyse der Implementierung der Navigationssysteme oder der Vergleich der Kosten verschiedener Technologien wird nicht durchgeführt. 
    Die Arbeit bleibt auf die technischen und methodischen Aspekte der Navigation beschränkt.

  \item Analyse anderer Sensorik: 
    Diese Arbeit untersucht ausschließlich bildbasierte Methoden und schließt die Analyse anderer Sensortechnologien wie \gls{LIDAR}, Radar oder Ultraschall bewusst aus.

  \item GPS-Verfügbarkeit: 
    Diese Arbeit geht von einer kontinuierlich funktionierenden \ac{GPS}-Verfügbarkeit aus und untersucht nicht die Navigationsmethoden in \gls{GPS-denied} Umgebungen. 
    Szenarien, in denen das \ac{GPS}-Signal gestört oder nicht verfügbar ist, werden daher bewusst nicht betrachtet.

  \item Energieverbrauch: 
    Der Energieverbrauch der Navigationssysteme wird nicht analysiert, da die Arbeit sich auf die Effizienz und Robustheit der Navigation selbst konzentriert und nicht auf die Energieeffizienz der Systeme im Gelände.
\end{itemize}

