\section{Aktueller Stand der Forschung}

Die theoretischen Grundlagen für bildbasierte Navigationstechnologien haben sich in den letzten Jahren stark entwickelt. 
Bildverarbeitung als zentrale Technologie für die autonome Navigation hat signifikante Fortschritte gemacht und wird zunehmend in verschiedenen Anwendungen eingesetzt \cite{image:processing:uav:autonomous,multi:objective:ugv:navigation}. 
Verschiedene Algorithmen, wie \ac{CNN} und Optical Flow, werden erfolgreich eingesetzt, um Umgebungsdaten in Echtzeit zu verarbeiten und daraus Navigationsentscheidungen abzuleiten \cite{autonomous:flight:uwb:positioning,mapless:ugv:navigation}. 
Der Verzicht auf \gls{LIDAR} zugunsten bildbasierter Systeme ermöglicht kostengünstigere und flexiblere Lösungen, stellt jedoch höhere Anforderungen an die Genauigkeit und Robustheit der Bildverarbeitung \cite{website:tesla:suppport}.

Diese Arbeit baut auf bestehenden Forschungsergebnissen auf und zielt darauf ab, die spezifischen Herausforderungen und Möglichkeiten bildbasierter Navigation in schwierigen Geländetypen zu untersuchen \cite{mapless:ugv:navigation,ugv:coverage:energy:efficient}. 
AirSim, eine hochdetaillierte visuelle und physikalische Simulationsplattform, wird dabei eingesetzt, um realistische Umgebungen zu simulieren und die Effizienz und Robustheit der Navigationsalgorithmen zu bewerten \cite{airsim2017fsr}.

