\section{Risikoanalyse}

Die Bildverarbeitung in unwegsamem Gelände oder bei schlechten Lichtverhältnissen stellt eine signifikante Herausforderung dar, die die Effizienz und Zuverlässigkeit der Navigation beeinträchtigen kann \cite{mapless:ugv:navigation}. 
Ein zentrales Risiko besteht darin, dass die Algorithmen aufgrund der Komplexität der Umgebung fehlerhafte Interpretationen der Bilddaten liefern könnten. 
Dies kann zu falschen Wegpunkten führen, was im schlimmsten Fall eine ineffiziente Navigation oder sogar gefährliche Situationen für das \ac{UGV} zur Folge haben könnte \cite{ugv:coverage:energy:efficient}.
    
Ein weiteres erhebliches Risiko ist die potenzielle Unzuverlässigkeit der Algorithmen bei stark variierenden Umgebungsbedingungen. Veränderungen wie plötzlicher Wetterumschwung, unvorhersehbare Hindernisse oder variable Lichtverhältnisse können die Leistung der Bildverarbeitungsalgorithmen erheblich beeinträchtigen \cite{autonomous:flight:uwb:positioning}. 
Diese Unwägbarkeiten könnten zu einer verminderten Präzision der Wegpunktgenerierung führen, was die Gesamteffizienz der Navigation negativ beeinflussen könnte.

Zur Minimierung dieser Risiken werden die Bildverarbeitungsergebnisse kontinuierlich überwacht, um frühzeitig Abweichungen oder Fehler in der Wegpunktgenerierung zu erkennen. 
Bei Bedarf werden die Algorithmen angepasst, um ihre Leistung unter den gegebenen Bedingungen zu optimieren \cite{image:processing:uav:autonomous}.
