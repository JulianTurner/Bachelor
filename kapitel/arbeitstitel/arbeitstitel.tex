\section{Mögliche Arbeitstitel}

Der derzeitige Arbeitstitel dieser Bachelorarbeit lautet: 
Untersuchung der Navigation eines autonomen Fahrzeugs im Gelände: Ein Experiment mit bildbasierter Routenführung.

Dieser Titel spiegelt die zentralen Elemente der Untersuchung wider:
\begin{itemize}
    \item Navigation
    \item Autonomes Fahrzeug
    \item Gelände
    \item Bildbasierte Technologien
\end{itemize}

Im Folgenden werden alternative Titel vorgestellt, die unterschiedliche Schwerpunkte der Arbeit betonen:

\begin{itemize}
    \item Bildbasierte Navigation für autonome Fahrzeuge im Gelände: Ein Vergleich von eigenständiger und UAV-unterstützter Wegfindung\\
    Dieser Titel legt den Fokus auf den Vergleich zweier Navigationsansätze: 
    die selbstständige Navigation des Fahrzeugs und die UAV-unterstützte Wegfindung. 
    Der Schwerpunkt liegt auf den Unterschieden dieser bildbasierten Methoden.

    \item Geländenavigation für autonome Fahrzeuge: Ein Experiment mit bildverarbeitungsbasierten Ansätzen\\
    Hier steht die Navigation im Gelände im Vordergrund, wobei das Experiment und die Analyse bildbasierter Methoden im Zentrum der Untersuchung stehen.

    \item Bildverarbeitung in der autonomen Navigation: Ein Experiment mit eigenständigen und UAV-unterstützten Methoden\\
    Dieser Titel betont die Rolle der Bildverarbeitung als zentrale Technologie der Arbeit und fokussiert den Vergleich zwischen eigenständiger und UAV-unterstützter Navigation.
\end{itemize}

Diese Titelvarianten reflektieren unterschiedliche Schwerpunkte der Arbeit und bieten jeweils einen spezifischen Blickwinkel auf die Forschung. 
Die endgültige Wahl des Titels wird davon abhängen, welche Aspekte der Untersuchung als besonders relevant und erkenntnisreich herausgestellt werden sollen. 
Die Arbeit zielt darauf ab, einen Beitrag zur Weiterentwicklung der autonomen Fahrzeugtechnologie zu leisten, insbesondere im Hinblick auf die Optimierung von Navigationsmethoden in komplexen Geländeszenarien.

