\section{Zielsetzung}

Das Ziel dieser Arbeit ist es, die Effizienz und Robustheit bildbasierter Navigationsmethoden für ein \ac{UGV} zu untersuchen. 
Dabei werden zwei Ansätze verglichen:

\begin{enumerate}
    \item 
    Das Fahrzeug erhält lediglich einen Zielpunkt und findet den Weg selbstständig anhand der Bilder, die von den am \ac{UGV} angebrachten Kameras erfasst werden. 
    Hierbei wird der Videostream des \ac{UGV} ausgewertet, um die Route in Echtzeit zu generieren. 
    \item 
    Ein \ac{UAV} fliegt zum Zielpunkt, untersucht das Gelände mittels Bildverarbeitung in der \gls{BirdView}-Perspektive und übergibt \ac{UGV} eine Liste von Wegpunkten. 
\end{enumerate}

Ein weiterer Fokus liegt darauf, die Flexibilität und Anpassungsfähigkeit dieser Ansätze in verschiedenen Geländetypen zu bewerten. 
Für das Experiment wird Python, \gls{OpenCV}, \gls{AirSim} und die \gls{Unreal Engine} verwendet.
