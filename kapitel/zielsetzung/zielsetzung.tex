\section{Zielsetzung}

Das Ziel dieser Arbeit ist es, die Effizienz und Robustheit bildbasierter Navigationsmethoden für ein \ac{UGV} zu untersuchen. Dabei werden zwei Ansätze miteinander verglichen:

\begin{enumerate}
    \item Eigenständige Navigation: 
    Das Fahrzeug erhält lediglich einen Zielpunkt und findet den Weg selbstständig anhand der Bilder, die von den am \ac{UGV} angebrachten Kameras erfasst werden. 
    Der Videostream des \ac{UGV} wird in Echtzeit ausgewertet, um die optimale Route zu generieren.
    \item UAV-unterstützte Navigation: 
    Ein \ac{UAV} fliegt zum Zielpunkt, untersucht das Gelände aus der \gls{BirdView}-Perspektive mittels Bildverarbeitung und übergibt dem \ac{UGV} eine Liste von Wegpunkten zur Navigation. 
    Das \ac{UGV} folgt diesen Wegpunkten, um das Ziel zu erreichen.
\end{enumerate}

Ein weiterer Schwerpunkt der Arbeit liegt auf der Bewertung der Flexibilität und Anpassungsfähigkeit dieser Ansätze in unterschiedlichen Geländetypen. 
Die Untersuchung wird unter Verwendung von Python, \gls{OpenCV}, \gls{AirSim} und der \gls{Unreal Engine} durchgeführt.

