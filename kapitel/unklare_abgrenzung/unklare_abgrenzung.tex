\section{Unklare Abgrenzung und Forschungslücke}
In der bisherigen Forschung wurden zahlreiche Ansätze zur autonomen Navigation in verschiedenen Umgebungen untersucht. 
Dabei wurden sowohl bildbasierte Navigationsmethoden als auch UAV-unterstützte Systeme erforscht, jedoch fehlt es an einer umfassenden und direkten Gegenüberstellung dieser beiden Ansätze. 
Die meisten Studien konzentrieren sich entweder auf die Vorteile von UAVs bei der Pfadplanung oder auf die Leistungsfähigkeit von eigenständigen bildbasierten Navigationssystemen, ohne jedoch die Effizienz und Robustheit beider Ansätze unter vergleichbaren Bedingungen zu analysieren \cite{multi:objective:ugv:navigation,ugv:coverage:energy:efficient}.

Diese Arbeit adressiert diese Forschungslücke, indem sie eine systematische Analyse und einen direkten Vergleich zwischen eigenständigen bildbasierten Navigationsmethoden und UAV-unterstützten Ansätzen durchführt. Insbesondere wird untersucht, wie sich diese Methoden in realitätsnahen, schwierigen Geländetypen verhalten und welche Vor- und Nachteile sich in Bezug auf Effizienz, Flexibilität und Robustheit ergeben \cite{mapless:ugv:navigation,airsim2017fsr}. 
Durch diesen Vergleich soll ein tieferes Verständnis für die Einsatzmöglichkeiten und Grenzen der beiden Ansätze gewonnen werden, was bisher in der Forschung nur unzureichend behandelt wurde.
