\section{Vorläufige Gliederung}

\begin{enumerate}
    \item Einleitung
    \begin{itemize}
        \item Hintergrund und Relevanz des Themas
        \item Problemstellung und Forschungslücke
        \item Zielsetzung und Fragestellungen
    \end{itemize}
    
    \item Aktueller Stand der Forschung
    \begin{itemize}
        \item Entwicklungen in der bildbasierten Navigation
        \item Vergleich von bildbasierten und \gls{LIDAR}-Methoden
        \item Relevante Studien zur UAV-gestützten Wegpunktnavigation
        \item Forschungslücken und offene Fragen
    \end{itemize}
    
    \item Theoretische Grundlagen
    \begin{itemize}
        \item Grundlagen der Bildverarbeitung in autonomen Systemen
        \item UAV-gestützte Navigation mittels Bildverarbeitung
        \item Wegpunktnavigation und Pfadplanung
        \item Einsatz von Simulationsumgebungen wie \gls{AirSim} für autonome Systeme
        \item Anwendung von \gls{OpenCV} für die Wegpunkterstellung und -analyse
    \end{itemize}
    
    \item Methodik
    \begin{itemize}
        \item Topic Modelling nach Lutz und Lutz
        \item Beschreibung der Simulationsumgebung \gls{AirSim}
        \item \gls{OpenCV}-basierte Wegpunkterstellung und Integration in die Navigation
        \item Vergleich der Navigationsmethoden (autonom vs. UAV-unterstützt)
        \item Risikomanagement und Herausforderungen in der Navigation
    \end{itemize}
    
    \item Analyse der Navigationsansätze
    \begin{itemize}
        \item Zeit- und Genauigkeitsanalyse der Wegpunkterstellung
        \item Verhalten und Robustheit in verschiedenen Geländetypen
        \item Auswertung der Effizienz und Robustheit der Navigationsmethoden
    \end{itemize}
    
    \item Diskussion der Ergebnisse
    \begin{itemize}
        \item Vergleich der Navigationsansätze
        \item Interpretation der Ergebnisse im Kontext bestehender Forschung
        \item Implikationen für den praktischen Einsatz
    \end{itemize}
    
    \item Schlussfolgerung und Ausblick
    \begin{itemize}
        \item Zusammenfassung der wichtigsten Ergebnisse
        \item Empfehlungen für zukünftige Forschungsarbeiten
        \item Potenzielle Weiterentwicklungen und Anwendungen
    \end{itemize}
\end{enumerate}

