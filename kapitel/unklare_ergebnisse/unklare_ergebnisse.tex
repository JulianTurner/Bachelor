\section{Unklare Ergebnisse und Ihre Präzisierung}

Um sowohl theoretische als auch praktische Implikationen für die bildbasierte Navigation von autonomen Fahrzeugen im Gelände zu erzielen, ist es entscheidend, klare und verwertbare Ergebnisse zu generieren. 
Diese Arbeit zielt darauf ab, detaillierte Erkenntnisse darüber zu liefern, wie verschiedene Navigationsansätze unter unterschiedlichen Bedingungen abschneiden und welche spezifischen Vor- und Nachteile sie bieten. 

Allerdings besteht die Möglichkeit, dass einige Ergebnisse aufgrund der Komplexität der getesteten Szenarien oder der variablen Umgebungsbedingungen nicht eindeutig sind. 
Unklare Ergebnisse könnten beispielsweise auftreten, wenn die Leistung der Navigationsalgorithmen in bestimmten Szenarien stark variiert oder wenn es keine signifikanten Unterschiede zwischen den getesteten Methoden gibt. 
Solche Unklarheiten könnten die Interpretation der Ergebnisse und die Ableitung von Empfehlungen erschweren.

Um diese potenziellen Unklarheiten zu präzisieren, werden folgende Ansätze verfolgt:

\begin{itemize}
    \item Erweiterte Datenerhebung: Bei unklaren Ergebnissen wird die Datenerhebung erweitert, um zusätzliche Informationen zu sammeln, die zur Klärung beitragen könnten. Dies könnte durch die Durchführung weiterer Iterationen in der Simulation oder durch die Erhebung zusätzlicher Metriken geschehen, die bisher unberücksichtigt blieben.
    
    \item Detaillierte statistische Analyse: Es wird eine detaillierte statistische Analyse der Ergebnisse durchgeführt, um auch subtile Unterschiede und Trends erkennen zu können. Dabei werden statistische Methoden eingesetzt, um die Varianz der Ergebnisse zu analysieren und möglicherweise versteckte Muster zu identifizieren.
    
    \item Qualitative Bewertung: In Fällen, in denen quantitative Daten nicht ausreichen, wird eine qualitative Bewertung der Ergebnisse vorgenommen. Dies umfasst eine genauere Untersuchung der spezifischen Szenarien, in denen Unklarheiten aufgetreten sind, sowie eine Diskussion möglicher Ursachen und Auswirkungen.
\end{itemize}

Durch diese Maßnahmen wird angestrebt, Unklarheiten zu reduzieren und fundierte, belastbare Ergebnisse zu erzielen, die klare Empfehlungen für den Einsatz bildbasierter Navigation in verschiedenen Anwendungsszenarien ermöglichen.
