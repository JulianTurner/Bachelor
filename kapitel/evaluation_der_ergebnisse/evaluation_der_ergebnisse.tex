\section{Evaluation der Ergebnisse-}

Die Ergebnisse werden anhand von Metriken wie Navigationszeit, Genauigkeit der Wegpunktverfolgung und Ausweichverhalten gegenüber Hindernissen bewertet. 
Diese Metriken werden genutzt, um die Effizienz und Robustheit der beiden Ansätze zu quantifizieren.


\section{Evaluation der Ergebnisse}

Die Evaluation der Ergebnisse erfolgt anhand klar definierter Metriken, die die Leistung der untersuchten Navigationsansätze umfassend bewerten. 
Im Mittelpunkt stehen dabei folgende Metriken:

\begin{itemize}
    \item Navigationszeit: Diese Metrik misst die Zeit, die das \ac{UGV} benötigt, um einen definierten Zielpunkt zu erreichen. Kürzere Navigationszeiten deuten auf effizientere Navigationsansätze hin. Diese Metrik ist besonders relevant, um die Gesamtzeit der Routenplanung und -durchführung zwischen den verschiedenen Methoden zu vergleichen.
    
    \item Genauigkeit der Wegpunktverfolgung: Diese Metrik bewertet, wie präzise das \ac{UGV} den vorgegebenen Wegpunkten folgt. Abweichungen von den idealen Wegpunkten werden gemessen, um die Genauigkeit der Bildverarbeitung und der Navigationsalgorithmen zu beurteilen. Eine höhere Genauigkeit ist ein Indikator für die Verlässlichkeit des Navigationsansatzes.
    
    \item Ausweichverhalten gegenüber Hindernissen: Diese Metrik analysiert, wie effektiv das \ac{UGV} auf unerwartete Hindernisse reagiert und ob es in der Lage ist, sicher und effizient auszuweichen. Hierbei wird sowohl die Fähigkeit zur Hinderniserkennung als auch die Reaktionszeit bewertet. Ein gutes Ausweichverhalten ist entscheidend für die Robustheit der Navigation in realen Einsatzszenarien.
\end{itemize}

Die Kombination dieser Metriken ermöglicht eine umfassende Bewertung der Effizienz und Robustheit der untersuchten Navigationsansätze. 
Die Ergebnisse werden statistisch ausgewertet, um signifikante Unterschiede zwischen den Methoden zu identifizieren und fundierte Empfehlungen für den Einsatz in verschiedenen Anwendungsszenarien abzuleiten. 
Dabei wird auch die Varianz der Ergebnisse analysiert, um die Zuverlässigkeit und Konsistenz der Ansätze zu bewerten.
