\section{Einleitung}
Autonome Fahrzeuge, die in der Lage sind, komplexes Gelände autonom zu durchqueren, stehen vor der Herausforderung, 
effektive Navigationsmethoden zu nutzen. 
Insbesondere ist unklar, ob ein autonomes Fahrzeug eigenständig den besten Weg zu einem Zielpunkt finden 
sollte oder ob es besser wäre, ein \ac{UAV} zur Unterstützung einzusetzen, 
das die Umgebung erkundet und eine Liste von Wegpunkten an das Fahrzeug übermittelt.

Junge Fahrzeughersteller, wie Tesla, setzen zunehmend auf bildbasierte Systeme, 
die Kameras und fortschrittliche Computer-Vision-Algorithmen nutzen, um das Umfeld zu erfassen und zu navigieren.
Diese Arbeit konzentriert sich auf die Untersuchung der Effizienz und Robustheit von Navigationsmethoden, 
die ausschließlich auf Bildverarbeitung basieren. 
Diese Forschung wird im Rahmen eines größeren Forschungsumfelds durchgeführt, 
das auch das \ac{IMUGS}-Projekt umfasst, obwohl die Arbeit selbst nicht direkt in das Projekt integriert ist.