% \section{Einleitung}
% Autonome Fahrzeuge, die in der Lage sind, komplexes Gelände autonom zu durchqueren, stehen vor der Herausforderung, 
% effektive Navigationsmethoden zu nutzen. 
% Insbesondere ist unklar, ob ein autonomes Fahrzeug eigenständig den besten Weg zu einem Zielpunkt finden 
% sollte oder ob es besser wäre, ein \ac{UAV} zur Unterstützung einzusetzen, 
% das die Umgebung erkundet und eine Liste von Wegpunkten an das Fahrzeug übermittelt.
%
% Tesla begann 2019 damit \gls{LIDAR} für die Navigation zu entfernen und setzt zunehmend auf bildbasierte Systeme, 
% die Kameras und fortschrittliche Computer-Vision-Algorithmen nutzen, um das Umfeld zu erfassen und zu navigieren \}.
% Diese Arbeit konzentriert sich auf die Untersuchung der Effizienz und Robustheit von Navigationsmethoden, 
% die ausschließlich auf Bildverarbeitung basieren. 
% Diese Forschung wird im Rahmen eines größeren Forschungsumfelds durchgeführt, 
% das auch das \ac{IMUGS}-Projekt umfasst, obwohl die Arbeit selbst nicht direkt in das Projekt integriert ist.
%

\section{Einleitung}

\ac{UGV} spielen eine immer wichtigere Rolle in verschiedenen Anwendungsbereichen, insbesondere in der Verteidigung, Katastrophenhilfe und industriellen Automatisierung. 
Ihre Fähigkeit, komplexes und gefährliches Gelände autonom zu durchqueren, macht sie zu unverzichtbaren Werkzeugen in Szenarien, in denen menschliche Intervention schwierig oder gefährlich ist \cite{uav:ugv:search:rescue}. 
\ac{UGV} bieten dabei nicht nur operative Vorteile, sondern tragen auch zur Sicherheit und Effizienz in kritischen Situationen bei \cite{ugv:system:indonesia}. 
Angesichts der wachsenden Herausforderungen, denen Europa in sicherheitsrelevanten Bereichen gegenübersteht, hat die Europäische Union den Bedarf erkannt, die Entwicklung solcher Systeme voranzutreiben. 
Dies spiegelt sich unter anderem im \ac{IMUGS}-Projekt wider, das darauf abzielt, modulare und skalierbare unbemannte Bodensysteme zu entwickeln, die den spezifischen Anforderungen der EU-Mitgliedstaaten gerecht werden \cite{mil:ugv:attitudes}.
Die Navigation von \ac{UGV} in unwegsamem Gelände stellt jedoch eine erhebliche technische Herausforderung dar. 
Es ist insbesondere unklar, ob ein autonomes Fahrzeug eigenständig den optimalen Weg zu einem Zielpunkt finden sollte oder ob es effizienter wäre, ein \ac{UAV} zur Unterstützung einzusetzen, das die Umgebung erkundet und eine Liste von Wegpunkten an das Fahrzeug übermittelt \cite{ugv:coverage:energy:efficient}. 
Diese Fragestellung ist von entscheidender Bedeutung, da die Wahl der Navigationsmethode direkten Einfluss auf die Einsatzfähigkeit und Zuverlässigkeit von \ac{UGV} hat \cite{ugv:uav:cooperative:ranging}.
In den letzten Jahren haben führende Fahrzeughersteller, wie Tesla, begonnen, \gls{LIDAR} für die Navigation zu entfernen und stattdessen zunehmend auf bildbasierte Systeme zu setzen \cite{website:tesla:suppport}. 
Diese Systeme nutzen Kameras und fortschrittliche Computer-Vision-Algorithmen, um das Umfeld zu erfassen und zu navigieren \cite{website:tesla:suppport}. 
Bildverarbeitungstechnologien bieten eine kosteneffiziente Alternative zu \gls{LIDAR}, erfordern jedoch hochpräzise und robuste Algorithmen, um in realen Einsatzszenarien zuverlässig zu funktionieren \cite{mil:ugv:attitudes}.
Diese Arbeit konzentriert sich auf die Untersuchung der Effizienz und Robustheit von Navigationsmethoden, die ausschließlich auf Bildverarbeitung basieren. 
Die Forschung wird im Rahmen eines größeren Forschungsumfelds durchgeführt, das auch das \ac{IMUGS}-Projekt umfasst, obwohl die Arbeit selbst nicht direkt in das Projekt integriert ist. 
Ziel dieser Arbeit ist es, fundierte Erkenntnisse darüber zu gewinnen, wie verschiedene bildbasierte Navigationsansätze in unterschiedlichen Geländearten abschneiden und welche Methoden sich für den Einsatz in zukünftigen \ac{UGV}-Systemen besonders eignen.

